\documentclass[12pt,a4paper,titlepage]{article}
\usepackage{fullpage}
\usepackage{hyperref}
\usepackage[pdftex]{graphicx}
\author{
  Mavrommatis, Marinos\\ 
\texttt{mm1g10@ecs.soton.ac.uk}
\and
Petroaica, Alex\\ 
\texttt{first2.last2@xxxxx.com}
\and
Karkallis, Panicos\\ 
\texttt{pk1g10@ecs.soton.ac.uk}
\and
Perez-Mavrogenis, Dionisio\\
 \texttt{first2.last2@xxxxx.com}
\and
Zardilis, Argyris\\
 \texttt{az2g10@ecs.soton.ac.uk}
\and
Svensson, Kim\\
 \texttt{ks6g10@ecs.soton.ac.uk}
\and
Oikonomou, Alexandros\\
 \texttt{ao2g10@ecs.soton.ac.uk}
}
\title{Cool title}

\begin{document}
\maketitle

\section{Description of prototype functionality}
\subsection{Background}
The Internet, since its inception, has been a basis for information and knowledge sharing. With the advent of the Web in the last two decades it has also become increasingly interactive. This increasing amount of interactivity started out with Usenet newsgroups, the IRC and discussion groups which were (and still are) very popular especially among technology people. The ease of use of web tools and the widespread adoption of the PC and the World Wide Web has made the average user able to interact easily through the Web and the last few years transformed him from a mere consumer of information to a consumer/publisher of information through the use of blogs, social networking sites.  This led to the creation of themed online-communities allowing people to interact and share knowledge on just about any topic. Especially popular though are technology related communities such as discussion forums for a specific topic(ubuntu users etc.) or question-answer based websites (stack overflow).

We have found, through our university experience, that we rely heavily on user-generated information sharing websites such as Wikipedia, blogs or question-answer websites(which often involve discussion) for our work. Also the use of social networking sites for communication(primarily) and information sharing in the University environment -either among group members for a specific coursework, students in a specific course or society- has become a de facto standard. Our aim was the creation of a web application targeted at students of our department (ECS) that combined characteristics of question-answer based websites(best answer, user reputation), the interactivity offered by Facebook groups for communication and information sharing and also some form module exploration(module-specific information that do not show up at the Syllabus) for lower-year students that would assist in the module selection process.
\subsection{Prototype}
Our application tries to bridge the gap between information sharing websites and Facebook groups by providing a way to do both through a forum-like interface.  Registered users can subscribe to different modules to see module-specific posts. They can also create their own threads for discussion/question asking or reply to threads created by other users. Threads and replies can be up-voted and down-voted by users. Highly rated replies/threads as shown first or higher in the presentation of a thread.  Users with a big number of highly rated posts get a higher reputation(karma).

The functionality of the application is extended by allowing the users to rate modules, courseworks and lecturers. Users can also add their grade for the module thus creating module specific statistics which provide a body of knowledge passed from year to year, information that would otherwise be difficult to obtain and which facilitate module selection. 
\newpage
\section{Tools and techniques used}
 \begin{description}
  \item[Emacs, vim, notepad++] \hfill \\
  These text editors were used for writing code.(you don't say)
  \item[Git] \hfill \\
  Git is a distributed version control and source-code management system. Developing with version control has many advantages especially in collaborative projects. It allows for tracking of changes per user, reverting back to previous versions, branching to test experimental features. 
  \item[Github] \hfill \\
  Github is a free and open-source web-based project hosting service that uses the Git version control system.  It provides a nice graphical interface to git repositories by allowing you to see coloured diffs, repository history, changes by user, relevant statistics of contribution, discussion by commit comments. It also provides hosting so it can be used as a central copy(master) of the repository for backup and reference. It also provides 'wiki' functionality for each project.
\item[Facebook] \hfill \\
  A facebook group was created for internal communication and discussion between members of the group. 
\item[Google Docs] \hfill \\
  Google docs were used for more formal and persistent communication of ideas, tasks to be done and requirements. It allows for collaborative live editing, a feature that proved very useful.
\end{description}
Also techniques used in the project include:
\newpage
\section{Relevant statistics}
\begin{table}[!htbp]
\centering
\begin{tabular}{| c |c |c|}
\hline
\bf{Language/File type} & \bf{Non-blank lines}\\
\hline
Python & 1553\\
\hline
Javascript & 581 \\
\hline
css & 1029 \\
\hline
html & 1311\\
\hline
\bf{Total} & \bf{4474}\\
\hline
\end{tabular}
\caption{Lines of code for project per language}
\end{table}
External tools used:
\newpage
\section{Overview of design and implementation}
\newpage
\section{Critical evaluation}
\end{document}