\documentclass[12pt,a4paper,titlepage]{article}
\usepackage{fullpage}
\usepackage{hyperref}
\usepackage[pdftex]{graphicx}
\author{
  Mavrommatis, Marinos\\ 
\texttt{mm1g10@ecs.soton.ac.uk}
\and
Petroaica, Alex\\ 
\texttt{first2.last2@xxxxx.com}
\and
Karkallis, Panicos\\ 
\texttt{pk1g10@ecs.soton.ac.uk}
\and
Perez-Mavrogenis, Dionisio\\
 \texttt{first2.last2@xxxxx.com}
\and
Zardilis, Argyris\\
 \texttt{az2g10@ecs.soton.ac.uk}
\and
Svensson, Kim\\
 \texttt{ks6g10@ecs.soton.ac.uk}
\and
Oikonomou, Alexandros\\
 \texttt{ao2g10@ecs.soton.ac.uk}
}
\title{Cool title}

\begin{document}
\maketitle

\section{Description of prototype functionality}
\subsection{Background}
The Internet, since its inception, has been a basis for information and knowldege sharing. With the advent of the Web in the last two decades it has also become increasingly interactive. This increasing amount of interactivity started out with Usenet newsgroups, the IRC and discussion groups which were (and still are) very popular especially among technology people. The ease of use of web tools and the widespread adoption of the PC and the World Wide Web has made the average user able to interact easily through the Web and the last few years tranformed him from a mere consumer of information to a consumer/publisher of information through the use of blogs, social networking sites.  This led to the creation of themed online-communities allowing people to interact and share knowledge on just about any topic. Especially popular though are technology related communities such as discussion forums for a specific topic(ubuntu users etc.) or question-answer based webstites (stack overflow).

We have found, through our university experience, that we rely heavily on user-generated information sharing websites such as Wikipedia, blogs or question-answer websites(which often involve discussion) for our work. Also the use of social networking sites for communication(primarily) and information sharing in the University environment -either among group members for a specific coursework, students in a specific course or society- has become a de facto standard. Our aim was the creation of a web application targeted at students of our department (ECS) tha combined characteristics of question-answer based websites(best answer, user reputation), the interactivity offered by Facebook groups for communication and information sharing and also some form module exploration(module-specific information that do not show up at the Syllabus) for lower-year students that would assist in the module selection process.
\subsection{Prototype}


\section{Tools and techniques used}
\section{Relevant statistics}
\section{Overview of design and implementation}
\section{Critical evaluation}
\end{document}